\documentclass{article}

\usepackage[utf8]{inputenc}
\usepackage{url}

\begin{document}

\title{Travail pratique \#3 - IFT-3065}
\author{Nicolas Lafond et Gevrai Jodoin-Tremblay}
\date{Dimanche 30 Avril 2017}
\maketitle

\section{Introduction}
Le but de ce travail était d'apprendre comment générer du code \emph{C} à partir
du langage fonctionnel \emph{Typer}. Pour ce faire nous avons transformer la
forme intermédiaire \emph{elexp} vers une nouvelle forme intermédiaire appelé
\emph{cexp}. Cette forme nous était fourni avec le travail, toutefois nous avons
décider de faire quelques changements sur cette forme afin de pouvoir
représenter les fermetures plus facilement dans le code généré.

\section{Structure du programme}
\subsection{compile.ml}
Nous avons décidé de créer un nouveau module \emph{OCaml} pour le compilateur
vers \emph{C}.
Nous avons ajouté la règle \emph{typerc} dans le makefile du projet pour compiler
le compilateur vers \emph{C} au lieu de l'interpréteur \emph{Typer}. Le fichier
\emph{compile.ml}
reprend une partie du code du fichier \emph{REPL.ml} mais au lieu d'évaluer
les expressions, il les transforment en \emph{cexp} puis génère le 
code \emph{C}. Ainsi,
à l'image des exécutables \emph{ocaml} vs \emph{ocamlc}, nous pouvons compiler
un programme \emph{typer} vers \emph{C} grâce à l'exécutable \emph{typerc}.
Aussi, bien que par défaut le fichier \emph{C} généré soit nommé \emph{a.c},
on peut changé celui-ci grâce aux arguments \emph{--output} ou \emph{-o}.

\subsection{cexp.ml}
Éffectue la transformation d'un \texttt{(vname*elexp) list list} vers un
\emph{cfile}, y compris le hoisting et la conversion de fermetures des lambdas.

\subsection{codify.ml}
Écriture du code \emph{C} correspondant à un \emph{cfile}. Pour s'assurer de ne
pas avoir de conflits entre les noms de variables et fonctions de \emph{Typer}
et celles de \emph{runtime\_support.c}, les variables \emph{Typer} sont toutes
préfixées d'un \emph{underscore} \texttt{\_} dans le code, contrairement au code
de \emph{runtime\_support.c}.


\section{Conversion de fermetures}
Nous avons fait la conversion de fermetures tel que vu dans le cours dans le
passage de \emph{elexp} vers \emph{cexp}. Nous avons ajouter un type
\emph{Closure} dans \emph{cexp} pour représenter un constructeur de fermetures.
Chaque définition de fonctions dans le fichier source \emph{Typer} est alors
transformé en construction de fermetures dans la fonction \texttt{main} du
programme \emph{C} généré.

Lors de la conversion de \emph{elexp}, nous effectuons directement une
conversion des lambdas en fermetures. Ainsi, dès que l'on trouve un lambda dans
l'arbre de \emph{elexp}, nous commençons par chercher toutes les variables
libres de son corps. \cite{closureconversion} Nous avons donc une liste ordonnée
de variables, et nous pouvons simplement repasser au travers du corps du lambda
pour changer toutes les variables se trouvant dans le tableau par l'indice dans
ce tableau. Aussi, puisque nous effectuons cette transformation en
\emph{depth first}, les variables internes qui auraient cachées les variables
globales ont déjà changé de noms lorsque vient le temps de capturer celles-ci.
Ensuite, nous faisons simplement \emph{hoister} ce lambda dans un tableau global
(mutable, pas très fonctionnel, nous savons) en lui donnant un nom unique.
Finalement, une fermeture \emph{Cexp.Closure} remplace ce lambda, celui ci
prenant le nom du lambda (soit le nom de la fonction qui sera généré en
\emph{C}) qu'il ferme ainsi que le tableau de ces variables libres.

\section{Génération de code C}
Puisque notre préoccupation principale étant de générer du code \emph{C} qui
compile sous \emph{GCC} de la manière la plus simple possible, nous ne nous
somme pas souciés de la lisibilité du code généré. Ainsi le code \emph{C}
généré ne resemble pas à du code qui aurait été écrit par un humain. Le code
généré à la forme suivante:
\begin{enumerate}
\item Inclusion du fichier de \textit{runtime\_support}
\item Déclaration de toutes les lambdas et les variables globales
\item Définitions des fonctions (qui sont toutes les fonctions fermées que
  nous avons obtenues en faisant la conversion de fermetures sur les fonctions
  du programme \emph{typer})
\item La fonction principale \textit{main} qui va construire toutes variables
  et fermetures et qui correspond à l'exécution du programme.
\end{enumerate}

\subsection{Type des données primitives}
Puisque les expressions \emph{cexp} n'ont aucunes informations sur les types
des expressions (ceux-ci étant vérifiés et effacés pendant la phase de 
transformation de \emph{lexp} vers \emph{elexp}) alors nous avions besoins d'un
type générique pour les déclarations dans le code C. Nous avons d'abord pensé à
utiliser des pointeurs génériques de type \emph{void*} puis après avoir discuté
avec plusieurs autres étudiants du cours nous avons conclus qu'une
représentation des types par une \emph{union} comprenant tous les types de bases
possibles serait plus appropriée. Inclure les \emph{Closures} dans cette
\emph{union} nous aide à traiter les fonctions comme toute autre variables.
Ce type union contient chacun des types primitifs de \emph{Typer} soit:
les entiers, les nombres à virgules flottantes, les booléens, les chaînes de
caractères et les fermetures(pour les fonctions). Une bonne chose est que nous
pouvions assumer qu'il n'y a pas d'erreurs de typage dans les expressions
\emph{elexp} que nous reçevions, donc nous n'avions pas à nous préoccuper de
vérifier que les expressions sont bien typé dans le code généré.

\subsection{Librairie de runtime\_support}
En plus de la représentation des types primitifs, nous avons également créé
des fonctions pour construire et appeller les fermetures, ainsi que pour
construire tous les différents types primitifs.
Finalement, nous avons implémenté des fonctions équivalentes aux fonctions
\textit{builtins} pour ensuite transformer des fonctions \textit{builtins}
de \emph{Typer} vers celle équivalentes en \emph{C}.

\subsection{Génération des \emph{let}}
Nous avons cherché assez longtemps pour trouver une façon de représenter en
C une suite d'expressions dont la dernière seulement doit être retourné, pour
finalement trouver une extension de \emph{GCC} permettant exactement ceci,
soit les \emph{statement expressions}\cite{gnu_bracedgroup}
\cite{stack_bracedgroup}. Cette extension propre
à \emph{GCC} est normalement utilisé principalement dans la définition de
\emph{Macros}, mais convient parfaitement à notre problème. En présumant
n'utiliser que \emph{GCC} pour compiler le code généré, ceci ne devrait pas
causer de problèmes.

Puisque le variables déclarées à l'intérieur de cette expression n'entrent
pas en conflits avec les variables externes (\emph{shadowing} géré
automatiquement), et que la valeur de retour d'un de ces blocs d'instructions
est évalué presque comme une variable ou une fonction
classique (on peut voir ces \emph{compounds} un peu comme une fonction 
\emph{inline}, ceci facilite grandement la tâche de compilation. En effet,
l'immense avantage de cette forme est nous pouvons maintenant traiter les
\emph{lets} comme n'importe qu'elle autre expression. Ainsi, tout ce que
nous faisons, c'est imprimer une suite de
\texttt{(\{primtype 'letvar' = cexp ; (\{ ... ;
  'expression à retourner' \})...\})}
représentant toutes les déclarations. La dernière expression est ainsi évalué
comme la valeur de l'expression complète.


\subsection{Représentation des datatypes}
Nous avons pensé à représenter les datatypes par des unions qualifié tel que
mentionné sur le site web officiel de OCaml, il s'agit de structures composé
d'un entier pour représenter le type du datatype et d'une union pour
représenter les valeurs du datatypes.

\section{Aide externe}
Il est a mentionner que nous avons été aidé par François Dufour et Vincent
Amyong qui sont aussi dans le cours. Cependant, cette aide ne fût que pour aider
à notre compréhension initiale du travail, principalement pour la conversion de
fermetures et la capture de variables libres.

De plus, nous avons consulté le travail de Simon Bernier St-Pierre (aussi dans
le cours) sur \emph{Github} \cite{sbstp} pour tenter de comprendre la capture de
variables avec le \emph{runtime\_env} défini dans le fichier \emph{env.ml}.
Nous sommes conscient que ceci n'est pas idéal, mais soyez par contre assuré
que cette consultation n'était que pour aider à notre compréhension, et que
l'implémentation est la nôtre, bien qu'un peu similaire.
 
\begin{thebibliography}{9}

\bibitem{realworldocaml}
	Yaron Minsky, Anil Madhavapeddy and Jason Hickey,
	Real World OCaml,
	\url{http://realworldocaml.org},
	O'Reilly,
	2012.

\bibitem{ocaml}
	Learn OCaml
	\url{https://ocaml.org/learn/},

\bibitem{inria}
	Xavier Leroy et al.,
	OCaml Doc and user's manual,
	\url{http://caml.inria.fr/pub/docs/manual-ocaml},
	Institut National de Recherche en Informatique et en Automatique,
	2013.

\bibitem{closureconversion}
	Matt Might,
  Closure conversion: How to compile lambda
	\url{http://matt.might.net/articles/closure-conversion/},
  University of Utah.
  
\bibitem{modern compiler implementation in ML}
  Andrew W. Appel,
  modern compiler implementation in ML,
  1998.

\bibitem{gnu\_statexprs}
  GNU GCC Online documentation: Statements and Declarations in Expressions,
	\url{https://gcc.gnu.org/onlinedocs/gcc/Statements-Exprs.html}.

\bibitem{stack\_bracedgroup}
  Are compound statements (block) surrounded by parens expressions in Ansi C?,
  \url{stackoverflow.com/questions/1238016}.

\bibitem{sbstp}
  Simon Bernier St-Pierre,
  Ift3065/tp3,
	\url{https://github.com/sbstp/ift3065},
  2017.
  

\end{thebibliography}
\end{document}
